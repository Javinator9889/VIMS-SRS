Este documento se estructura de la siguiente manera:

\begin{enumerate}
  \item En la sección \ref{sec:description} se define la descripción general del
        producto. Allí, se abordan aspectos directamente relacionados con el producto en sí
        como son definirlo brevemente junto con el contexto de desarrollo y
        la perspectiva del mismo (a quién va orientado, qué funcionalidades tendrá)
        (\ref{ssec:perspective}), qué características deben tener o se contemplan en
        los usuarios finales (\ref{ssec:specs}), qué restricciones de diseño, técnicas
        o regulaciones se aplican sobre el producto (\ref{ssec:restrictions}) y,
        finalmente, aquellas suposiciones y dependencias sobre el producto que se
        asumirán a lo largo del proyecto.
  \item En la sección \ref{sec:requirements} se especifican formalmente los
        requisitos del proyecto. En primer lugar, se definen aquellos requisitos
        de usuario que especifican las necesidades finales de las personas con respecto
        del producto (\ref{ssec:user-req}). En segundo lugar, se exponen los requisitos
        funcionales a un nivel más técnico e interno. Estos requisitos definen la
        base del producto y aquellas funcionalidades que deben existir en el resultado
        final (\ref{ssec:functional-req}). Por otra parte, se continúa detallando
        los requisitos no funcionales. Estos requisitos si bien no añaden funcionalidad
        al sistema directamente, limitan y acotan el alcance del mismo (\ref{ssec:non-functional-req}).
        Allí se incluye la especificación de las interfaces externas
        del sistema, es decir, aquellos elementos externos con los que interactuará
        (\ref{ssec:external-if-req}); restricciones a aplicar durante el desarrollo del
        producto, en donde se tienen en cuenta estándares, limitaciones físicas, etc
        (\ref{ssec:dev-restrictions}); y por último, los requisitos que se imponen del
        entorno físico del producto, que limitan y definen bajo qué circunstancias
        debe funcionar y bajo cuáles no (\ref{ssec:phisical-req}).
  \item Al final de la especificación, en los anexos, se incluyen los distintos diagramas
        que complementan la especificación, como casos de uso, bloques, etc. (\ref{chap:diagrams}).
\end{enumerate}