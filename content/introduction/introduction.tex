En un mundo cada vez más interconectado, hay ciertas tecnologías que se quedan
por detrás en unos campos mientras que siguen progresando en otros. Esto se ve
directamente reflejado en la industria automovilística en donde los vehículos
cada vez cuentan con mayor y mejor tecnología (como cámaras, sensores, actuadores,
etc.) pero no es directamente accesible por el usuario: mediante pantallas e
interfaces se ofrecen métodos sencillos que facilitan su uso.

\ac{VIMS} pretende ser un sistema que facilite el acceso a todos los datos que
ofrece un vehículo para generar estadísticas, descubrir patrones en la conducción
y detectar errores. De esta forma, el conductor tendrá información de primera
mano sobre el estado de su vehículo, eficiencia de su conducción así como obtener
información en tiempo real complementaria a la ya propiciada por el vehículo.

En este documento se realiza la especificación de requisitos y se acota el
alcance del proyecto, en donde se define de forma clara y concisa qué va a hacer
el sistema y qué no va a hacer.