\begin{acronym}
  \acro{GB}{\textit{gigabyte}}
  \acro{IoT}{\textit{Internet of Things}}
  \acro{PFM}{Proyecto Fin de Máster}
  \acro{VIMS}{\textit{Vehicle IoT Metrics System}}
  \acro{OBD}{\textit{On-Borad Diagnostics}}
  \acro{GPS}{\textit{Global Positioning System}}
  \acro{BLE}{\textit{Bluetooth Low Energy}}
  \acro{PAN}{\textit{Personal Area Network}}
  \acro{CAN}{\textit{Controller Area Network}}
  \acro{API}{\textit{Application Programming Interface}}
  \acro{PAN}{\textit{Personal Area Network}}
\end{acronym}

\begin{itemize}
  \item \ac{GB} -- unidad de almacenamiento de información equivalente a $10^9$ bytes.
  \item \ac{IoT} -- concepto que se refiere a la interconexión digital de objetos 
        cotidianos con Internet \cite{InternetCosas2021}.
  \item \ac{OBD} -- sistema de diagnóstico a bordo de vehículos que
        cuenta con múltiples estándares según la región de uso. Estos
        sistemas ofrecen una monitorización activa y control completo
        sobre el motor y otros dispositivos del vehículo \cite{OBD2021}.
  \item \textit{jitter} -- retardo relativo que se produce en las comunicaciones
        y que afecta directamente a la saturación de la red y a la capacidad de
        transmisión de la misma.
  \item \ac{GPS} -- sistema que permite posicionar cualquier objeto con una 
        precisión de hasta centímetros usando cuatro o más satélites y 
        trilateración \cite{GPS2021}.
  \item trilateración -- método matemático que permite determinar las posiciones
        relativas de objetos usando la geometría de los triángulos \cite{Trilateracion2021}.
  \item \ac{BLE} -- tecnología \ac{PAN} que permite la comunicación entre dispositivos
        con un rango similar a Bluetooth pero un menor consumo de energía.
  \item Bus \ac{CAN} -- protocolo de comunicaciones para el envío de
        mensajes en entornos distribuidos, permitiendo la comunicación
        entre múltiples CPUs.
  \item \ac{API} -- conjunto de definiciones, subrutinas y protocolos que ofrecen
        ciertos \textit{softwares} para ser usados por otras aplicaciones como
        capa de abstracción sobre el original \cite{InterfazProgramacionAplicaciones2021}.
  \item \ac{PAN} -- redes destinadas a la comunicación entre dispositivos en una
        misma red o malla. Tiene un alcance muy limitado, de unos pocos metros por
        lo general.
\end{itemize}