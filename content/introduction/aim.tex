El objetivo principal de este \ac{PFM} es el de diseñar un sistema de recolección
de métricas orientado a vehículos en el ámbito del \ac{IoT} (\ac{VIMS}, de ahora en adelante)
que permita, utilizando los conectores estándar del vehículo, generar y almacenar
información relevante del vehículo, la conducción y demás factores de interés que
se den mientras se interactúa con el vehículo.

El sistema \ac{VIMS} deberá poder conectarse a Internet desde cualquier punto
geográfico\footnote{Dentro de las restricciones y limitaciones físicas y 
geográficas del entorno.} usando el automóvil como sistema de alimentación y
fuente de información. De esta manera, podrá enviar todo tipo de datos provistos 
a un servidor en donde se gestionarán, almacenarán y procesarán para 
una posterior visualización y generación de información.

Además, el sistema deberá poder integrarse con cualquier vehículo del mercado que
utilice las conexiones estándar reguladas y que trabaje con tramas e información
estandarizada. En otro caso, el sistema no funcionará correctamente y puede
comportarse de manera impredecible.

La infraestructura del servidor por su parte deberá poder recibir una gran cantidad
de datos (según estimaciones del mundo \ac{IoT}, se pueden recibir del orden de
varios \ac{GB} diarios \cite{vishHowMuchData2020}) y gestionarlos debidamente.
Cada dispositivo emisor se considerará único, por lo que los datos recibidos
deberán ser clasificados acorde a quién los emite.

Sin embargo, el sistema no modificará parámetros del vehículo ni realizará
modificaciones sobre la configuración del mismo: se limitará a ser un ``espía''
y no emitirá ningún dato hacia el automóvil.

Finalmente, desde el propio vehículo con un dispositivo externo (como un
\textit{smartphone}) se podrán acceder a los datos en tiempo real que ofrece el
vehículo mediante una interfaz hacia el sistema \ac{VIMS}. De esta manera, se
podrá saber rápidamente el estado del automóvil y detectar fallos en el mismo.

Así, el producto está orientado para su implantación en cualquier vehículo y que,
haciendo uso de las características de conectividad inalámbrica que presentará, pueda
adaptarse a nuevos automóviles y nuevas tramas estándar. Su venta está dirigida
principalmente a conductores que cuenten con un vehículo con algún conector estándar
compatible.

Este documento sigue la taxonomía de especificación descrita en la sección anterior (\ref{sec:purpose})
y abraca una actividad adicional que consistirá en una validación de los requisitos,
definida en el anexo \ref{chap:validation}.