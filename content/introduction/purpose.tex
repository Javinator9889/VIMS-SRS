El propósito de este documento es múltiple: por una parte, el de establecer
un punto de partida claro y conciso en el desarrollo del proyecto. Una correcta
especificación acompañada de sus correspondientes diagramas
permite iniciar el diseño, desarrollo y verificación del proyecto con ideas
claras y evaluadas con anterioridad, afrontando los posibles fallos o problemas
que puedan surgir y previendo situaciones complejas o errores en el diseño.

Por otra parte, el documento se redacta como especificación de
requisitos de usuario, funcionales, no funcionales, restricciones en el
desarrollo, requisitos de interfaces externas y de entorno físico del proyecto.
Esto sirve como ``contrato'' con aquello que debe aparecer y existir en
una implementación final del proyecto (\textit{requisitos funcionales}),
otros requisitos que son importantes pero que, en un momento dado, no aportan
funcionalidad al producto (\textit{requisitos no funcionales}), las
características de los usuarios que quieran usar el producto y otras restricciones
o información relevante que se haya de tener en cuenta a la hora de diseñar, 
desarrollar y probar el producto antes de darlo por concluido.

Finalmente, este documento se redacta orientado a otros ingenieros que
tengan curiosidad o interés en cómo se ha desarrollado el proyecto, las necesidades
que se han de suplir, qué características conforman el producto final o mismamente
replicar el proyecto para hacer una implementación propia o añadir alguna 
característica nueva. Sin embargo, con intención de facilitar la accesibilidad del
documento, se buscará ser claro y conciso en la especificación, usando un
lenguaje que sea preciso. Por ello, se incluye un apartado de definiciones,
acrónimos y abreviaturas (sección \ref{sec:definitions}) que pretenden servir
de orientación en el lenguaje técnico que se usará en el documento.