\subsubsection{Precisión}
\begin{enumerate}[label=\textbf{\texttt{RNF-\arabic*}}]
  \item\label{nf:accuracy} El módulo de geolocalización ofrecerá una precisión
  que vendrá limitada directamente por el sistema utilizado. Así, con el GPS se
  puede llegar a obtener una precisión media de 5 metros mientras que con redes
  móviles la precisión media está en torno a los 60 metros. Por ello, no se considera algo crítico.
\end{enumerate}

\subsubsection{Rendimiento}
\begin{enumerate}[resume, label=\textbf{\texttt{RNF-\arabic*}}]
  \item\label{nf:read-speed} El sistema deberá leer y recibir los datos del
  conector estándar a la máxima velocidad que permita este.
  \item\label{nf:transmission-speed} El sistema enviará los datos a medida los
  vaya recibiendo. Si hay congestión en la red y los datos no se pueden enviar tan
  rápido como se generan, se almacenarán en un \textit{buffer} hasta poder transmitirlos.
  \item\label{nf:rt-viewer} La aplicación móvil deberá recibir directamente los
  datos desde el dispositivo \ac{VIMS} con un retardo mínimo de $\numprint[ms]{200}$
  desde su petición, indiferentemente de la tasa de refresco.
\end{enumerate}


\subsubsection{Eficiencia}
\begin{enumerate}[resume, label=\textbf{\texttt{RNF-\arabic*}}]
  \item\label{nf:battery-duration} Según la sección \ref{sec:restrictions},
  se prevé que el dispositivo pueda funcionar sin alimentación directa del coche
  cuando queden datos por enviar. Por razones de seguridad, el tiempo máximo que
  el sistema permanecerá activo será de hasta una hora. Así se evita drenar la
  batería del coche. Pasado ese periodo, los datos que no se pudieron transmitir
  se almacenarán de forma definitiva en memoria persistente hasta que el vehículo
  se encienda de nuevo.
  \item\label{nf:network-election} A razón del requisito \ref{nf:battery-duration},
  se deberá escoger la red priorizando aquella que suponga un menor consumo. Este
  criterio se puede basar en usar la que ya esté activa (ya que el inicio de una
  interfaz de red conlleva un alto consumo), usar una red de menor potencia o
  usar la red que esté disponible. En otro caso, se almacenarán los datos pendientes
  de envío en \ac{VIMS}.
\end{enumerate}

\subsubsection{Mantenibilidad}
\begin{enumerate}[resume, label=\textbf{\texttt{RNF-\arabic*}}]
  \item\label{nf:solid} Los componentes \textit{software} estarán estructurados
  siguiendo patrones de código limpio (como los principios S.O.L.I.D.) de forma
  que permitan una fácil mantenibilidad del mismo.
  \item\label{nf:ota-time} Las actualizaciones OTA no deberán tardar más de
  5 minutos en instalarse (se obvia el tiempo de descarga debido a que se depende
  directamente de la red en la que se esté conectado), para evitar un alto periodo
  de \textit{downtime}.
\end{enumerate}

\subsubsection{Portabilidad}
\begin{enumerate}[resume, label=\textbf{\texttt{RNF-\arabic*}}]
  \item\label{nf:easy-conn} Como se mencionó en la perspectiva del producto (\ref{sec:perspective}),
  el sistema debe ser fácilmente montable y desmontable, permitiendo el uso directo
  del conector estándar para otras labores.
\end{enumerate}

\subsubsection{Recuperabilidad}
\begin{enumerate}[resume, label=\textbf{\texttt{RNF-\arabic*}}]
  \item\label{nf:restart-time} En caso de fallo inesperado del sistema, este
  deberá reiniciarse en no más de 30 segundos, si el vehículo está activo. En otro
  caso, permanecerá desactivado.
\end{enumerate}

\subsubsection{Seguridad}
\begin{enumerate}[resume, label=\textbf{\texttt{RNF-\arabic*}}]
  \item\label{nf:err-mobile} Si no se detecta ningún tipo de red, el sistema
  avisará al usuario de dicho evento y se almacenarán los datos temporalmente hasta
  que esté disponible.
  \item\label{nf:err-app} Si la aplicación no detecta conexión con el servidor,
  notificará al usuario de dicho evento.
  \item\label{nf:err-gps} Dependiendo de la precisión de la ubicación del GPS,
  el sistema informará al usuario de la calidad del mismo y de las ``consecuencias''
  (falta de precisión en el recorrido, en el aparcamiento, etc.).
\end{enumerate}

\subsubsection{Integridad}
\begin{enumerate}[resume, label=\textbf{\texttt{RNF-\arabic*}}]
  \item\label{nf:enc} Los mensajes enviados hacia y desde el servidor estarán siempre
  cifrados y debidamente protegidos.
  \item\label{nf:ota-int} Las actualizaciones de la placa deberán estar verificadas
  y protegidas, para evitar instalar \textit{malware} de un atacante.
  \item\label{nf:s2a-int} Las comunicaciones directas con la aplicación deberán
  ser previamente verificadas, para evitar enviar información a otro(s) dispositivo(s)
  no autorizados.
  \item\label{nf:auth} Antes de acceder a ningún tipo de dato producido por \ac{VIMS},
  el usuario deberá autenticarse apropiadamente.
\end{enumerate}

\subsubsection{Escalabilidad}
\begin{enumerate}[resume, label=\textbf{\texttt{RNF-\arabic*}}]
  \item\label{nf:server-scalability} El servidor web puede recibir cientos (o miles)
  de peticiones cada minuto, por lo que deben tomarse medidas específicas que faciliten
  la escalabilidad del mismo.
\end{enumerate}

\subsubsection{Usabilidad}
\begin{enumerate}[resume, label=\textbf{\texttt{RNF-\arabic*}}]
  \item\label{nf:s-data} El sistema indicará distintos estados y mensajes mediante
  indicadores visuales, como luces LED.
  \item\label{nf:led-bright} Los indicadores visuales que se integren podrán ser ajustables
  por el usuario mediante un potenciómetro, que permitirá regular la intensidad lumínica
  a gusto del conductor.
\end{enumerate}
