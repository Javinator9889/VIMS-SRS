\subsubsection{Precisión}
\begin{enumerate}[label=\textbf{\texttt{RNF-\arabic*}}]
  \item\label{nf:accuracy} El módulo de geolocalización deberá ofrecer una precisión elevada,
  de entre 5 y 10 metros. Sin embargo, no se considera algo crítico (ver requisito
  \ref{nf:net-acc}).
  \item\label{nf:net-acc} Si el
  módulo \ac{GPS} no se encuentra accesible se usarán las redes móviles para
  obtener una ubicación aproximada. El rango máximo permitido está
  entre 5 y 50 metros. Por ello, la precisión de la ubicación no es un factor
  crítico ya que se permite bastante margen.
\end{enumerate}

\subsubsection{Rendimiento}
\begin{enumerate}[resume, label=\textbf{\texttt{RNF-\arabic*}}]
  \item\label{nf:read-speed} El sistema deberá leer y recibir los datos del
  conector estándar a la máxima velocidad que permita este. Así, si por
  ejemplo el conector ofrece 100 kbps, \ac{VIMS} deberá leer a esa velocidad.
  Sin embargo, con la intención de acotar este campo, se establece el máximo
  de velocidad aceptable en hasta 200 kbps.
  \item\label{nf:transmission-speed} El sistema enviará los datos a medida los
  vaya recibiendo. Si hay congestión en la red, los almacenará en un \textit{buffer}
  para un posterior envío.
  \item\label{nf:rt-viewer} La aplicación móvil deberá recibir directamente los
  datos desde el dispositivo \ac{VIMS} con una velocidad máxima de 200 kbps, como
  se define en \ref{nf:read-speed}.
\end{enumerate}

\subsubsection{Disponibilidad}
\begin{enumerate}[resume, label=\textbf{\texttt{RNF-\arabic*}}]
  \item\label{nf:start} El sistema se iniciará automáticamente con el arranque
  del vehículo y permanecerá activo mientras esté proporcione corriente. Cuando
  se apague, podrá funcionar temporalmente mediante una fuente de alimentación
  externa para gestionar los datos que queden, según \ref{req:connectivity} y
  \ref{req:start-stop}.
\end{enumerate}


\subsubsection{Eficiencia}
\begin{enumerate}[resume, label=\textbf{\texttt{RNF-\arabic*}}]
  \item\label{nf:battery-duration} Según \ref{req:start-stop} y \ref{nf:start},
  se prevé que el dispositivo pueda funcionar sin alimentación directa del coche
  cuando queden datos por enviar. Se limita ese periodo de tiempo a un máximo de
  5 minutos, ya que se estima que la fuente de alimentación externa no será de
  gran capacidad.
  \item\label{nf:network-election} A razón del requisito \ref{nf:battery-duration},
  se deberá escoger la red priorizando aquella que suponga un menor consumo. Este
  criterio se puede basar en usar la que ya esté activa (ya que el inicio de una
  interfaz de red conlleva un alto consumo); usar una red de menor potencia; o
  usar la red que esté disponible. En otro caso, se almacenarán los datos pendientes
  de envío de manera eficiente en \ac{VIMS}.
\end{enumerate}

\subsubsection{Extensibilidad}
\begin{enumerate}[resume, label=\textbf{\texttt{RNF-\arabic*}}]
  \item\label{nf:components} De cara a futuras implementaciones, aquellas interfaces
  que puedan ofrecer conectividad con dispositivos externos (I$_2$C, UART, \dots)
  se expondrán para añadir nuevos módulos al dispositivo.
\end{enumerate}

\subsubsection{Interoperabilidad}
\begin{enumerate}[resume, label=\textbf{\texttt{RNF-\arabic*}}]
  \item\label{nf:s2server} El sistema \ac{VIMS} podrá enviar información al servidor
  remoto con los datos e información recogidos por el coche.
  \item\label{nf:s2a} El sistema \ac{VIMS} se podrá comunicar directamente con la
  aplicación móvil con los datos en tiempo casi real.
  \item\label{nf:a2server} La aplicación móvil podrá comunicarse con el servidor
  mediante la \ac{API} que se ha definido en \ref{req:api} para obtener información
  sobre trayectos pasados y el conductor, así como para mostrar notificaciones.
\end{enumerate}

\subsubsection{Mantenibilidad}
\begin{enumerate}[resume, label=\textbf{\texttt{RNF-\arabic*}}]
  \item\label{nf:solid} Los componentes \textit{software} estarán estructurados
  siguiendo patrones de código limpio (como los principios S.O.L.I.D.) de forma
  que permitan una fácil mantenibilidad del mismo.
  \item\label{nf:ota-time} Las actualizaciones OTA no deberán tardar más de 
  5 minutos en instalarse (se obvia el tiempo de descarga debido a que se depende
  directamente de la red en la que se esté conectado), para evitar un alto periodo
  de \textit{downtime}.
\end{enumerate}

\subsubsection{Portabilidad}
\begin{enumerate}[resume, label=\textbf{\texttt{RNF-\arabic*}}]
  \item\label{nf:easy-conn} Como se mencionó en la perspectiva del producto (\ref{sec:perspective}),
  el sistema debe ser fácilmente montable y desmontable, permitiendo el uso directo
  del conector estándar para otras labores.
  \item\label{nf:conn} El sistema deberá poder montarse fácilmente en cualquier
  vehículo que integre un conector estándar.
\end{enumerate}

\subsubsection{Recuperabilidad}
\begin{enumerate}[resume, label=\textbf{\texttt{RNF-\arabic*}}]
  \item\label{nf:restart-time} En caso de fallo inesperado del sistema, este
  deberá reiniciarse en no más de 30 segundos, si el vehículo está activo. En otro
  caso, permanecerá desactivado.
\end{enumerate}

\subsubsection{Confiabilidad}
\begin{enumerate}[resume, label=\textbf{\texttt{RNF-\arabic*}}]
  \item\label{nf:vehicle-data} El sistema deberá tener actualizados los parámetros
  del vehículo para generar datos reales y fiables.
  \item\label{nf:app-data} La aplicación deberá mostrar correctamente los datos
  recibidos con su correspondiente unidad.
  \item\label{nf:server-data} El servidor debe almacenar los datos respetando las
  marcas temporales con las que se recibieron, almacenarlos correctamente así como
  las estadísticas generadas.
\end{enumerate}

\subsubsection{Robustez}
\begin{enumerate}[resume, label=\textbf{\texttt{RNF-\arabic*}}]
  \item\label{nf:ota-strength} El sistema no podrá fallar cuando se esté realizando
  una actualización OTA.
\end{enumerate}

\subsubsection{Seguridad (\textit{safety})}
\begin{enumerate}[resume, label=\textbf{\texttt{RNF-\arabic*}}]
  \item\label{nf:err-mobile} Si no se detecta ningún tipo de red, el sistema 
  avisará al usuario de dicho evento y se almacenarán los datos temporalmente hasta
  que esté disponible.
  \item\label{nf:err-app} Si la aplicación no detecta conexión con el servidor,
  notificará al usuario de dicho evento.
  \item\label{nf:err-gps} Dependiendo de la precisión de la ubicación del GPS,
  el sistema informará al usuario de la calidad del mismo y de las ``consecuencias''
  (falta de precisión en el recorrido, en el aparcamiento, etc.).
\end{enumerate}

\subsubsection{Integridad}
\begin{enumerate}[resume, label=\textbf{\texttt{RNF-\arabic*}}]
  \item\label{nf:enc} Los mensajes enviados hacia y desde el servidor estarán siempre
  cifrados y debidamente protegidos.
  \item\label{nf:ota-int} Las actualizaciones de la placa deberán estar verificadas
  y protegidas, para evitar instalar \textit{malware} de un atacante.
  \item\label{nf:s2a-int} Las comunicaciones directas con la aplicación deberán
  ser previamente verificadas, para evitar enviar información a otro(s) dispositivo(s)
  no autorizados.
  \item\label{nf:auth} Antes de acceder a ningún tipo de dato producido por \ac{VIMS},
  el usuario deberá autenticarse propiamente.
\end{enumerate}

\subsubsection{Escalabilidad}
\begin{enumerate}[resume, label=\textbf{\texttt{RNF-\arabic*}}]
  \item\label{nf:server-scalability} El servidor web puede recibir cientos (o miles)
  de peticiones cada minuto, por lo que deberá escalar debidamente para poder gestionarlas.
\end{enumerate}

\subsubsection{Usabilidad}
\begin{enumerate}[resume, label=\textbf{\texttt{RNF-\arabic*}}]
  \item\label{nf:data} Los datos deberán ser fácilmente accesibles por el usuario
  tanto desde la aplicación como desde la interfaz web.
  \item\label{nf:s-data} El sistema indicará distintos estados y mensajes mediante
  indicadores visuales, como luces LED. Sin embargo, estos no deben ser especialmente
  brillantes o invasivos para evitar distracciones.
\end{enumerate}
