Analizando las características del producto y las necesidades del mercado, se
proponen los siguientes requisitos funcionales con respecto a diversas temáticas
y características de \ac{VIMS}:

\subsubsection{Conectividad}
\begin{enumerate}[label=\textbf{\texttt{RF-\arabic*}}]
  \item\label{req:connectivity} El sistema se conectará a cualquier vehículo que use
        un conector estándar, como el \ac{OBD}.
  \item\label{req:network} El sistema usará redes móviles para intentar
        estar conectado siempre a la red. De esta manera, se cumple la dependencia
        \ref{dep:internet} (pero se sigue sujeto a \ref{dep:connectivity}).

        Para esta labor, se propone realizar un uso combinado de redes móviles 4G/3G/2G,
        WiFi (si está disponible) y LoRa.
  \item\label{req:gps} El sistema deberá ser geolocalizable o bien mediante el uso
        de tecnologías GPS o bien mediante el uso de redes móviles.
  \item\label{req:bt} El sistema usará la conectividad Bluetooth para
        permitir la comunicación con dispositivos cercanos y mostrar información en
        tiempo casi real.
  \item\label{req:conf} El sistema ofrecerá una interfaz desde la cual se podrá
        configurar la cuenta asociada y el dispositivo en sí.
\end{enumerate}

\subsubsection{Datos}
\begin{enumerate}[resume, label=\textbf{\texttt{RF-\arabic*}}]
  \item\label{req:info} El sistema recogerá todo tipo de datos emitidos por el
        vehículo.
  \item\label{req:transmission} En relación con el requisito \ref{req:info},
        el sistema emitirá esos datos a un servicio en la red para su gestión.
  \item\label{req:storage} El sistema almacenará los datos en memoria hasta un máximo
        de 7 días, si no se han podido transmitir.
  \item\label{req:time} El sistema deberá llevar un control del tiempo para asociar
        el dato con la marca temporal en la que se obtuvo.
\end{enumerate}

\subsubsection{Servidor}
\begin{enumerate}[resume, label=\textbf{\texttt{RF-\arabic*}}]
  \item\label{req:reception} El servicio web recibirá los datos transmitidos por los
        múltiples sistemas \ac{VIMS} que existan.
  \item\label{req:management} El servicio web clasificará la información recibida
        por cada dispositivo y la asociará a la cuenta del usuario.
  \item\label{req:visualization} El servicio web ofrecerá una interfaz en donde
        el usuario podrá ver información relacionada con sus últimos viajes,
        estadísticas e información del vehículo.
  \item\label{req:api} El servicio web ofrecerá una \ac{API} para acceder
        desde otras máquinas a los datos almacenados e
        información estadística.
\end{enumerate}

\subsubsection{Usuario}
\begin{enumerate}[resume, label=\textbf{\texttt{RF-\arabic*}}]
  \item\label{req:register} El usuario se dará de alta en la plataforma y se
        asociará el dispositivo \ac{VIMS} a su cuenta.
  \item\label{req:rt} El usuario podrá usar su \textit{smartphone} para visualizar
        información en tiempo casi real de su vehículo.
\end{enumerate}

\subsubsection{Aplicación móvil}
\begin{enumerate}[resume, label=\textbf{\texttt{RF-\arabic*}}]
  \item\label{req:app-functions} La aplicación deberá poder conectarse de forma
        inalámbrica al sistema \ac{VIMS} en cuestión.
  \item\label{req:app-sampling} La aplicación deberá leer toda la información
        obtenida por \ac{VIMS} en tiempo casi real.
  \item\label{req:app-stats} La aplicación deberá poder mostrar información
        estadística en base a los datos almacenados en el servidor. Para ello,
        se hará uso del requisito \ref{req:api}.
\end{enumerate}