Analizando las características del producto y las necesidades del mercado, se
proponen los siguientes requisitos funcionales con respecto a diversas temáticas
y características de \ac{VIMS}:

\subsubsection{Conectividad}
\begin{enumerate}[label=\textbf{\texttt{RF-\arabic*}}]
  \item\label{req:connectivity} El sistema se conectará a cualquier vehículo que use
        un conector estándar, como el \ac{OBD}.
  \item\label{req:network} El sistema usará redes móviles para intentar
        estar conectado siempre a la red. De esta manera, se intenta cubrir la dependencia
        \ref{dep:internet} (pero se sigue sujeto a \ref{dep:connectivity}).

        Para esta labor, se propone realizar un uso combinado de redes móviles 4G/3G/2G
        y WiFi (si está disponible).
  \item\label{req:gps} El sistema deberá ser geolocalizable bien mediante el uso
        de tecnologías GPS o bien mediante el uso de redes móviles.
  \item\label{req:bt} El sistema usará redes \ac{PAN} para
        permitir la comunicación con dispositivos cercanos y mostrar información en
        tiempo casi real.
  \item\label{req:ota} El sistema, usando la conectividad de red, podrá actualizarse
        de forma remota para ofrecer nuevas funcionalidades.
\end{enumerate}

\subsubsection{Datos}
\begin{enumerate}[resume, label=\textbf{\texttt{RF-\arabic*}}]
  \item\label{req:info} El sistema recogerá todo tipo de datos emitidos por el
        vehículo y los clasificará según corresponda, con sus unidades de medida
        e información que representa.
  \item\label{req:transmission} En relación con el requisito \ref{req:info},
        el sistema emitirá esos datos a un servicio en la red para su gestión. Los
        datos deberán ir acompañados de una marca temporal que permita identificar
        el instante origen, como se especifica posteriormente en \ref{req:time}.
  \item\label{req:gps-data} Además de los datos del vehículo definidos en \ref{req:info},
        el sistema deberá enviar información sobre la geolocalización del vehículo
        en sí, según lo estipulado en \ref{req:transmission}.
  \item\label{req:storage} El sistema almacenará los datos en memoria si no se han podido transmitir
        hasta que o bien no quepan más datos o bien se hayan podido enviar. En caso
        de que la memoria estuviese completa, mediante indicadores visuales se le
        notificaría al usuario de este evento, dándole a entender que las entradas
        más antiguas serán eliminadas (ver \ref{nf:s-data}).
  \item\label{req:time} El sistema deberá llevar un control del tiempo para asociar
        el dato con la marca temporal en la que se obtuvo.
  \item\label{req:conf} El sistema ofrecerá una interfaz desde la cual se podrá
        configurar la cuenta asociada y el dispositivo en sí.
  \item\label{req:parking} El sistema, una vez se aparque el vehículo, deberá
        notificar su última posición a modo de recordatorio de dónde se ha aparcado.
  \item\label{req:stats} Las estadísticas de los viajes se generan tras un periodo
        de tiempo con el motor apagado. Dicho periodo se establece por
        defecto en 1 hora pero podrá ser configurable por el usuario desde la aplicación.
        Pasado ese tiempo, se generará la información relacionada con el
        trayecto realizado y se pondrá a disposición del usuario en forma de estadísticas
        y datos en sí. Opcionalmente, el usuario recibiría una notificación una vez
        estuviesen disponibles.
  \item\label{req:periodic-stats} Además de las estadísticas de los viajes,
        se generarán estadísticas temporales: cada semana, cada mes y cada año.
        En ellas, el usuario conocerá de primera mano el uso que ha hecho del coche,
        gasto estimado, desgastes, etc. y se le notificarán cuando estén disponibles.
\end{enumerate}

\subsubsection{Servidor}
\begin{enumerate}[resume, label=\textbf{\texttt{RF-\arabic*}}]
  \item\label{req:reception} El servicio web recibirá los datos transmitidos por los
        múltiples sistemas \ac{VIMS} que existan.
  \item\label{req:management} El servicio web clasificará la información recibida
        por cada dispositivo y la asociará a la cuenta del usuario correspondiente.
  \item\label{req:visualization} El servicio web ofrecerá una interfaz en donde
        el usuario podrá ver información relacionada con sus últimos viajes,
        estadísticas e información del vehículo.
  \item\label{req:api} El servicio web ofrecerá una \ac{API} para acceder
        desde otras máquinas a los datos almacenados e
        información estadística.
  \item\label{req:gps-follow} El servicio web mostrará información sobre la
        ubicación actual del coche, trazando un mapa con la ruta realizada.
  \item\label{req:server-parking} El servicio web mostrará al usuario la última
        ubicación conocida del vehículo, es decir, dónde está aparcado.
  \item\label{req:data-duration} El servidor almacenará los datos en principio de
        forma indefinida. Como todavía no se conoce la cantidad de información
        que se almacenará, en un futuro este requisito puede ser modificado y
        se puede establecer una temporalidad de los datos.
\end{enumerate}

\subsubsection{Usuario}
\begin{enumerate}[resume, label=\textbf{\texttt{RF-\arabic*}}]
  \item\label{req:register} El usuario se dará de alta en la plataforma y se
        asociará un dispositivo \ac{VIMS} a su cuenta.
  \item\label{req:rt} El usuario podrá usar su \textit{smartphone} para visualizar
        información en tiempo casi real de su vehículo.
  \item\label{req:maintenance} El usuario podrá definir información básica 
        sobre el estado del vehículo actualmente. De esta forma, tras el tiempo
        estipulado por el fabricante o tras cierto kilometraje este recibirá una
        notificación indicándole los mantenimientos a realizar.
  \item\label{req:driving} El sistema usará la información recogida según los
        requisitos \ref{req:info} y \ref{req:gps-data} para ofrecerle al usuario
        un perfil personalizado sobre su conducción. En dicho perfil se tendrán
        en cuenta los datos de revoluciones del motor, intensidad del acelerador,
        desnivel del terreno, consumo del vehículo, distancia recorrida y demás
        para adecuarlo bajo un perfil propio según diversos modelos existentes \cite{husseinaliameenDrivingBehaviourIdentification2021}.
\end{enumerate}

\subsubsection{Aplicación móvil}
\begin{enumerate}[resume, label=\textbf{\texttt{RF-\arabic*}}]
  \item\label{req:app-functions} La aplicación deberá poder conectarse de forma
        inalámbrica al sistema \ac{VIMS} en cuestión.
  \item\label{req:app-sampling} La aplicación deberá leer toda la información
        obtenida por \ac{VIMS} con retardo mínimo de hasta $\numprint[ms]{200}$.
  \item\label{req:app-stats} La aplicación deberá poder mostrar información
        estadística en base a los datos almacenados en el servidor. Para ello,
        se hará uso del requisito \ref{req:api}.
  \item\label{req:location} La aplicación podrá mostrar la última ubicación
        conocida del vehículo, es decir, dónde está aparcado.
\end{enumerate}