\subsubsection{Interfaces de usuario}
\begin{enumerate}[resume, label=\textbf{\texttt{RNF-\arabic*}}]
  \item\label{nf:pretty} Los datos mostrados en las interfaces de usuario deberán
  ser simples, claros y accesibles.
  \item\label{nf:devices} La interfaz de usuario estará distribuida en varios
  dispositivos: el servidor, la aplicación y \ac{VIMS} en sí. Desde las dos primeras,
  se ofrecen los datos recibidos en tiempo casi real, estadísticas, perfiles de
  conducción, etc. Desde el sistema, se muestra una interfaz básica de registro,
  configuración de la misma y gestiones básicas.
  \item\label{nf:initial-config} La configuración inicial de \ac{VIMS} se realizará
  o bien desde la placa en sí o bien desde la aplicación. Ambas interfaces serán
  similares.
  \item\label{nf:notifications} Las notificaciones al usuario en principio se
  realizan única y exclusivamente desde la aplicación. Sin embargo, en un futuro
  no se descarta enviar notificaciones también desde la interfaz web.
\end{enumerate}

\subsubsection{Interfaces \textit{hardware}}
\begin{enumerate}[resume, label=\textbf{\texttt{RNF-\arabic*}}]
  \item\label{nf:hardware-obd} La interfaz \textit{hardware} debe permitir al sistema
  conectarse con cualquier vehículo que equipe el conector estándar.
  \item\label{nf:hardware-net} La interfaz \textit{hardware} debe permitir al sistema
  la comunicación con el exterior por medio de redes móviles.
  \item\label{nf:hardware-gps} La interfaz \textit{hardware} debe permitir al sistema
  la geolocalización mediante redes móviles o \ac{GPS}.
\end{enumerate}

\subsubsection{Interfaces \textit{software}}
\begin{enumerate}[resume, label=\textbf{\texttt{RNF-\arabic*}}]
  \item\label{nf:protocol} La interfaz \textit{software} debe encapsular los
  datos enviados usando el protocolo que se decida emplear en posteriores fases
  de diseño.
  \item\label{nf:encryption} La interfaz \textit{software} debe poder encriptar
  los datos para su transmisión por la red.
  \item\label{nf:compression} La interfaz \textit{software} debe poder comprimir
  los datos para ahorrar durante su envío o almacenamiento.
\end{enumerate}

\subsubsection{Interfaces de comunicaciones}
\begin{enumerate}[resume, label=\textbf{\texttt{RNF-\arabic*}}]
  \item\label{nf:mobile} El sistema establecerá las comunicaciones con el servidor
  primeramente usando redes móviles, como 4G/3G/2G y LoRaWAN.
  \item\label{nf:wifi} El sistema establecerá las comunicaciones con el servidor
  como segunda opción mediante redes inalámbricas tipo LAN, como WiFi o Bluetooth
  (si están disponibles).
\end{enumerate}