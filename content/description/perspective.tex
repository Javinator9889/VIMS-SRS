\ac{VIMS} se constituirá de un módulo independiente diseñado desde cero aprovechando
las tecnologías que ofrecen los vehículos de forma estándar (como el conector
\ac{OBD}).

La intención principal es ofrecer un sistema de recolección de métricas autónomo,
automático y lo más simple posible para el usuario, que siga la idea de ``conectar y funcionar'',
con la intención de salvar la diferencia tecnológica existente entre vehículos
y otorgarle al conductor el control total.

Al igual que se narró en el alcance (\ref{sec:aim}), el objetivo principal del proyecto es
el de desarrollar un sistema autónomo que pueda funcionar con cualquier vehículo
del mercado (que cumpla con las condiciones especificadas) y que permita generar,
recolectar, procesar y mostrar cientos de datos relativos al coche y su estado
actual y estado pasado.

Para ello, se necesitará de una conexión permanente y activa a Internet (siempre
y cuando sea posible) por la cual se enviarán en flujo los datos del vehículo.
En posteriores etapas de diseño se valorará la cantidad de datos a enviar según
la red en uso, disponibilidad de los recursos, saturación local, \textit{jitter}
en las comunicaciones, etc.

La transmisión de los datos recibidos constituyen la cualidad característica
del sistema. Sin embargo, como es posible que por factores del entorno ciertos
valores no se puedan transmitir en el momento, estos se almacenarán en memoria
persistente (mínimo $\numprint[Mb]{1}$) hasta que haya una conexión de red
por la que enviarlos.

Por otra parte, se ofrece la posibilidad de ver los datos con retardo mínimo: dado que
no siempre es adecuado observar los datos \textit{a posteriori} sino que puede ser necesario evaluarlos en el momento, se podrá observar información en
el momento del estado del vehículo, información de sensores, etc. Esta
visualización se realizará mediante un dispositivo externo al sistema como
puede ser un \textit{smartphone}. Para facilitar la integración, se desarrollará
en paralelo una aplicación móvil específicamente diseñada para \ac{VIMS}.

Entre las demás características del producto, se contempla una más que es la
geolocalización del vehículo. Para ello, se usarán las distintas redes con
las que contará el producto así como el \ac{GPS}. De esta forma,
al usuario final no solo se le mostrarán estadísticas e información sobre sus
desplazamientos sino que también sabrá en qué puntos ha estado y así obtener más
información con respecto a su conducción. Esto permite también usar \ac{VIMS} como
medida de seguridad, en caso de hurto del vehículo, para saber su ubicación
precisa. También puede ser útil como recordatorio de dónde estaba aparcado el
coche, ya que solo habrá que revisar la última ubicación.

Finalmente, el sistema deberá integrarse de forma fácil y sencilla, y así debe ser
también su utilización. Esto se traduce en que tanto el
montaje como el desmontaje debe ser sencillo, permitiendo que si se necesita acceso
a los puertos estándar del vehículo el sistema no será un impedimento.

En definitiva, para conductores de vehículos que quieran conocer
más información sobre su automóvil, que necesiten hacer diagnósticos o que quieran
estadísticas/datos, \ac{VIMS} es un sistema autónomo integrado
que ayuda al conductor a generar y obtener los datos mencionados anteriormente. 
A diferencia de otras soluciones presentes
en el mercado, nuestro producto permitirá integrarse sin dificultades en cualquier
vehículo que cumpla con los estándares y sin ninguna configuración adicional para
el usuario. Además, será personalizable y podrá recibir actualizaciones
y mejoras que aumenten su funcionalidad y amplíen su compatibilidad 
con nuevos vehículos.