El producto en principio no está restringido en términos de legalidad, ya que las
actividades de monitorización que realiza en nada deberían
alterar la funcionalidad del vehículo al que se conectan 
(el producto, como se especificó en el alcance
-- \ref{sec:aim}, no está diseñado para modificar parámetros relativos al
vehículo).

Por otra parte, el sistema no solo se compone del dispositivo que va sobre el
automóvil en sí, sino también de la aplicación de visualización. Por
consiguiente, el desarrollo al completo deberá realizarse teniendo en cuenta
todos los componentes para que sea compatible desde el primer momento.

En lo referente a la presentación gráfica, los usuarios solicitaron que fuese
simple y accesible, así como personalizable. Se comentará más sobre esto en la
sección de requisitos de usuario (\ref{sec:user-req}), pero se comenta aquí ya
que es una limitación en el desarrollo en cuanto a que define una característica
que se considera necesaria.

Además, como el sistema se integra con los automóviles, la única fuente de
alimentación será el vehículo en sí. Existe una casuística en la que el conductor apaga el
coche pero existen datos que todavía no se han podido enviar. Ante esta situación,
como el conector \ac{OBD} sigue ofreciendo alimentación, se tiene que detectar este
evento y decidir si o bien se envían los datos pendientes si se cuenta con conexión de red o
bien se almacenan en memoria persistente hasta que el vehículo se encienda de
nuevo. De esta forma, no se compromete la batería del automóvil.

Como se ha contemplado anteriormente, el sistema será geolocalizable. Esto
conlleva tener en cuenta el consumo adicional de los módulos de geolocalización,
los cuales suelen tener asociados un elevado gasto energético, para evitar
un desgaste prematuro de la batería. Por otra parte, como la información recogida
del vehículo se puede considerar información sensible, debe permanecer privada y 
accesible únicamente al usuario poseedor de \ac{VIMS}.

Si el sistema funciona correctamente, se espera una implantación en el mercado
elevada y que se empiece a redistribuir de forma nacional. Esto se produciría
por la necesidad del mercado de este producto, la facilidad en su instalación y
su correcto funcionamiento final.