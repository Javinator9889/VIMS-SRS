El producto en principio no está restringido en términos de legalidad, ya que al
ser algo que se integra directamente con el vehículo no necesita cumplir con
ninguna regulación o legislación (el producto, como se especificó en el alcance
-- \ref{sec:aim}, no está diseñado para modificar parámetros relativos al
vehículo. Si sirviese para realizar tareas de \textit{tunning} debiera estar
regulado).

Por otra parte, el sistema no solo se compone del dispositivo que va sobre el
automóvil en sí sino también de la aplicación de visualización en tiempo casi
real y de la visualización de datos en forma de histórico o de estadísticas. Por
consiguiente, el desarrollo al completo deberá realizarse teniendo en cuenta
todos los componentes para que sea compatible desde el primer momento.

En lo referente a la presentación gráfica, los usuarios solicitaron que fuese
simple y accesible, así como personalizable. Se comentará más sobre esto en la
sección de requisitos de usuario (\ref{sec:user-req}), pero se comenta aquí ya
que es una limitación en el desarrollo en cuanto a que define una característica
que se considera necesaria por el usuario.

Además, como el sistema se integra con los automóviles, debe tomar alimentación de
ellos directamente. Existe una casuística en la que el conductor apaga el
coche pero existen datos que todavía no se han podido enviar y que se
perderían irremediablemente, por lo que es necesario tener en cuenta dicha
situación para sobrellevarla con, por ejemplo, una batería. Dicha batería
debe tener suficiente capacidad como para poder transmitir todas las tramas
restantes pero ser lo más pequeña posible para que no ocupe demasiado espacio.
Como su capacidad se estima baja,
\ac{VIMS} usará la energía proveniente de la batería única y exclusivamente
cuando se pierda la energía del vehículo.

Como se ha contemplado anteriormente, el sistema será geolocalizable. Esto
conlleva tener en cuenta el consumo adicional de los módulos de geolocalización
empleados a la hora de ejecutarse de forma independiente, para evitar un consumo en
exceso. Por otra parte, como la ubicación se puede considerar información sensible,
debe permanecer privada y accesible únicamente al usuario poseedor de \ac{VIMS}.

Si el sistema funciona correctamente, se espera una implantación en el mercado
elevada y que se empiece a redistribuir de forma nacional. Esto se produciría
por la necesidad del mercado de este producto, la facilidad en su instalación y
su correcto funcionamiento final.