\subsection*{Dependencias}
\begin{enumerate}[label=\textbf{\texttt{DEP-\arabic*}}]
  \item\label{dep:internet} El sistema necesitará siempre de una conexión a Internet para funcionar.
        En otro caso, almacenará los datos en la memoria persistente que se estima de al menos $\numprint[Mb]{1}$.
  \item\label{dep:connectivity} El sistema dependerá del área geográfica en la que se encuentre para enviar
        información, ya que pueden existir zonas en donde no haya ningún tipo de
        conectividad.
  \item\label{dep:gps} El sistema deberá localizarse al aire libre en una zona sin apantallar la señal de los satélites para
        ofrecer los servicios de geolocalización al completo.
  \item\label{dep:rt} Los dispositivos de visualización deberán encontrarse
        en un rango suficientemente cercano para realizar la transmisión de la
        información. Dependerá de la tecnología de red utilizada:
        WiFi, Bluetooth o \ac{BLE}.
  \item\label{dep:network} Las redes que se usen para transmitir datos pueden variar con el tiempo
        así como su disponibilidad y velocidad. El sistema debe estar preparado
        para este escenario y adecuarse correctamente.
  \item\label{dep:nt-speed} El método de transmisión y envío de datos por Internet hacia el servidor
        de gestión y almacenamiento debe ser lo más eficiente posible, ya que la
        calidad de la conexión puede ser mala (debido a \ref{dep:network}).
\end{enumerate}

\subsection*{Supuestos}
\begin{enumerate}[label=\textbf{\texttt{SUP-\arabic*}}]
  \item\label{sup:connectivity} Se supone que el vehículo en que se implante el sistema contará con un
        conector estándar que permita las comunicaciones y la alimentación del módulo,
        como el \ac{OBD}.
  \item\label{sup:server} Se supone que el servidor de recolección de datos será
        capaz de aguantar la demanda de los dispositivos destinados a pruebas y de
        una cantidad considerable de dispositivos en entorno de producción.
  \item\label{sup:uid} Se supone que cada \ac{VIMS} contará con un identificador
        único que permitirá identificar al dispositivo inequívocamente del resto.
  \item\label{sup:users} Se supone que cada usuario poseedor de un \ac{VIMS} contará
        con acceso a Internet recurrente y dispondrá de una cuenta en el servicio de
        estadísticas que le permita vincular su(s) dispositivo(s) \ac{VIMS} a su
        cuenta, según el \ref{sup:uid}.
  \item\label{sup:pids} Se supone que las tramas transmitidas por el vehículo
        son las estándar definidas de forma global para los automóviles del mercado.
        Este supuesto guarda una estrecha relación con \ref{sup:connectivity}.
\end{enumerate}